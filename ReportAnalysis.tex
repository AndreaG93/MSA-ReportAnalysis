\documentclass[sigchi]{acmart}
\usepackage{algorithm}
\usepackage[noend]{algpseudocode}
\usepackage{natbib}
\usepackage{quoting}

\AtBeginDocument{%
  \providecommand\BibTeX{{%
    \normalfont B\kern-0.5em{\scshape i\kern-0.25em b}\kern-0.8em\TeX}}}

\settopmatter{printacmref=false, printccs=false, printfolios =false}
\setcopyright{none} 
\renewcommand\footnotetextcopyrightpermission[1]{}

\acmConference[MSA A.A. 2019-2020]{ }{September, 2020}{ }

\begin{document}

\title{Analysis of "Cloudlet-based Efficient Data Collection in Wireless Body Area Networks" report}

\author{Andrea Graziani}
\email{andrea.graziani93@outlook.it}
\affiliation{%
  \institution{Università Degli Studi di Roma Tor Vergata}
  \city{Rome}
  \state{Italy}
}

\renewcommand{\shortauthors}{Andrea Graziani (0273395)}

\maketitle

\section{Introduction}

\vspace{0.3cm}

\begin{quoting}[font=itshape, begintext={``}, endtext={''\cite[par.~1.4]{MSAReport}}]
The main goal of this paper is to develop a large scale WBANs system in the presence of cloudlet-based data collection model. The objective is to minimize end-to-end packet cost by dynamically choosing data collection to the cloud by using cloudlet based system [...] reducing packet-to-cloud energy, the proposed work also attempt to minimize the end-to-end packet delay
\end{quoting}

\vspace{0.3cm}

According to \citet{MSAReport}, the goal of their work is to built an efficient \textbf{Wireless Body Area Networks} (\textbf{WBAN}) exploiting \textbf{edge computing}, a new paradigm in which substantial computing and storage resources, referred to as \textbf{cloudlets}, are placed at the Internet's edge, that is in close proximity to WBAN devices or sensors.\cite{TheEmergenceOfEdgeComputing}

To be more precise, as stated by \citet{MSAReport}, edge computing resources are exploited in order to \textbf{minimize average power consumption and delay} when transmitting collected data by Personal Digital Assistant (PDA) device to cloud. As we will see next, that intent is achieved through a \textit{self-adaptive behaviour} according to which \textbf{communication technology is dynamically changed} when users move from one region to another. 

\section{WBAN System architecture}

\vspace{0.3cm}

\begin{quoting}[font=itshape, begintext={``}, endtext={''\cite[par.~3.2]{MSAReport}}]
In our implementation, the enterprise cloud system will be the ultimate destination of the collected data. The enterprise cloud is also able to send messages back to both the cloudlet system or to WBAN users.
\end{quoting}

\vspace{0.3cm}

\begin{quoting}[font=itshape, begintext={``}, endtext={''\cite[par.~4.1]{MSAReport}}]
The enterprise cloud system is a centralized management and storage point that can be accessed by different organizations that are interested in a certain type of data. Another important feature of the cloudlet system is the ability of bidirectional communications between many WBANs users.
\end{quoting}

\vspace{0.3cm}

WBAN devices and sensors have very stringent constraints in term of CPU performance, storage resources and battery life to perform a wide range of healthcare, military and sport application. This is the reason because \textbf{Tier-1}, which represents "\textit{the cloud}" in today's parlance, plays a very important role in this kind of system: cloud resources are exploited to overcome WBAN devices penalty, since Tier-1 represents, in terms of archival preservation, the \textit{safest place to store data}, ensuring the long-term integrity and accessibility, and the \textit{main execution site to perform expensive computations}, thanks to its almost unlimited elasticity. 

However cloud resources exploitation isn't a panacea because there are some negative consequence, as the increasing of network \textbf{round-trip times} (\textbf{RTT}) experienced by mobile users; therefore, \citet{MSAReport} solution is to built a system capable to exploit edge computing resources, which, as it has shown, allow to offload compute-intensive operations at very low latency.\cite{TheSeminalRoleEdgeNativeApplications}\cite{TheEmergenceOfEdgeComputing}

To be more precise, WBAN system prototype modelled by \citet{MSAReport} is intended for \textit{edge-accelerated, cloud-native applications}. Why?

\begin{description}

\item[cloud-native] because, to overcome stringent constraints on battery, performance and storage of WBAN devices, cloud resources are exploited, through offloading techniques over a wireless network, \textbf{to execute critical task}; in other words, as stated also by \citet{MSAReport}, \textit{cloud is the primary execution site and it is essential to provide services to final users}.

\item[edge-accelerated] because the system exploits edge computing resources \textbf{only when available}. In that way, that system is capable to provide optimal performance, but \textit{edge resources represents a secondary execution site, that is they are optional to provide services}.

\end{description}

Unfortunately, it is shown edge-accelerated, cloud-native applications can \textbf{only partially} benefit from the advantages of edge computing in term of \textit{response times}, \textit{battery life}, \textit{ingress bandwidth demand reduction}, \textit{privacy} and \textit{fallback services}. However any application using this kind of system involves less investment risk, much less software development, and their markets are much larger since they can function acceptably even in the absence of edge computing.





















In fact, their study is based on the observation that, using WiFi technology, a WBAN user will be able to transmit data packet to the cloud with low power, low delay and mostly no connection cost compared with cellular technology:

\vspace{0.3cm}

\begin{quoting}[font=itshape, begintext={``}, endtext={''\cite[par.~3.1]{MSAReport}}]
It was shown that, via WiFi, the transmission power of a data packet of size 46 Bytes will cost about 30 mw and with a delay of 0.045 ms. On the other hand, a longer transmission range cellular network connection (e.g. 3G and LTE) is capable of transmitting the data packet to the cloud from any location that is cover by cellular network, which is usually a wider geographic area compared with the WiFi. It was shown that, via cellular, the transmission power of data packet of size 46 Bytes will cost about 300 mw and with a delay of 0.45 ms.
\end{quoting}

\vspace{0.3cm}

Obliviously, due both to the \textbf{need of edge infrastructure resources} and to \textbf{short transmission range} of WiFi technology compared to cellular network, cloudlet and WiFi coverage aren't available everywhere, therefore \citet{MSAReport} model their system identifying three different regions:\cite[par.~3.1]{MSAReport} 

\begin{description}

\item[Cloudlet Region (CR)] where WiFi coverage is available, so a user can use it to transmit data to the cloudlet.

\item[Enterprise Region (ER)] where only cellular coverage is available, therefore  a user can use only cellular technology to transmit a data packet to cloud. 

\item[Not-covered Region (NC)] where neither WiFi nor cellular technology is available. In this case a user should buffer the packets until one of the above technologies is available, then to be able of transmitting the packet to the enterprise cloud. 

\end{description}

\subsection{Architecture Overview}

\vspace{0.3cm}

\begin{quoting}[font=itshape, begintext={``}, endtext={''\cite[par.~3.2]{MSAReport}}]

The architecture of the enterprise system is mostly same architecture as in cloud based data centres with all its components and features. In our implementation, the enterprise cloud system will be the ultimate destination of the collected data. The enterprise cloud is also able to send messages back to both the cloudlet system or to WBAN users.

\end{quoting}

\vspace{0.3cm}

\begin{quoting}[font=itshape, begintext={``}, endtext={''\cite[par.~4.1]{MSAReport}}]

The enterprise cloud system is a centralized management and storage point that can be accessed by different organizations that are interested in a certain type of data. 

\end{quoting}

\vspace{0.3cm}

Based on what has been said in previous section, in \citet{MSAReport}'s system model WBAN users are capable to exploit edge computing resources if and only if they are in a \textit{Cloudlet Region}, otherwise they go on transmitting data directly to the cloud if possible; therefore, is evident that, as state by \citet{MSAReport} too, cloud plays a very important role in that system since represent the ultimate destination of collected data and perform critical task providing access to data. In other words, seem enough clear that cloudlet is less critical compared to cloud to providing system services.

Therefore, using \citet{EdgeNativeApplications} proposed application taxonomy, the WBAN system prototype modelled by \citet{MSAReport} is intended for \textit{edge-accelerated, cloud-native applications}:

\begin{description}

\item[cloud-native] because, to overcome stringent constraints on battery, performance and storage of WBAN devices, cloud resources are exploited, through offloading techniques over a wireless network, \textbf{to execute critical task}; in other words, cloud is the primary execution site and it is essential to provide services to final users.

\item[edge-accelerated] because the system exploits edge computing resources \textbf{only when available}. In that way, that system is capable to provide optimal performance, but edge resources represents a secondary execution site, that is they are optional to provide services.

\end{description}

Unfortunately, it is shown edge-accelerated, cloud-native applications can \textbf{only partially} benefit from the advantages of edge computing in term of \textit{response times}, \textit{battery life}, \textit{ingress bandwidth demand reduction}, \textit{privacy} and \textit{fallback services}. However any application using this kind of system involves less investment risk, much less software development, and their markets are much larger since they can function acceptably even in the absence of edge computing.

\section{Simulations analysis}

The performance of WBAN system proposed by \citet{MSAReport} is evaluated through several simulations focusing on two performance metrics, that is:

\vspace{0.3cm}

\begin{quoting}[font=itshape, begintext={``}, endtext={''\cite[par.~4.2]{MSAReport}}]
\textit{Packet Transmission Power} and \textit{Packet Delay}. The Packet Transmission Power and Packet Delay are directly measure of the communication energy and delay expenditure from the Personal Digital Assistant (PDA) device to the cloudlet or the enterprise cloud

\end{quoting}

\vspace{0.3cm}

Using a custom version of CloudSim simulator, according to \citet{MSAReport}, simulations were carried out monitoring a virtual space-area, where several cloudlet entities were deployed in known geographic locations in such a way transmission range belonging to different cloudlet not overlapping. Moreover, during simulations, several WBAN users were moved randomly with fixed speed and random pause time. 

\citet{MSAReport} performs several set of simulation varying several parameters of their system like:
\begin{itemize}
\item number of users.
\item number of deployed cloudlet.
\item number of Virtual Machines (VM) deployed in cloudlet.
\item area size.
\item cloudlet placement.
\item speed of processing a data packet on cloudlet. 
\end{itemize}

\subsubsection{First experiment set}

In order to observe the effect on data packet process delay and power consumption according to task load of the received data on the cloudlet, a first set of simulations are carried out \textbf{varying processing speed of data packet} between 100 and 900 million instructions per second (MIPS). 

In this first set of experiments, \citet{MSAReport} draw the following conclusions:

\begin{enumerate}

\item Fixed the average time required to process a packet data on a CPU, that is for a given MIPS process speed, power consumption and processing delay are increased by increasing the number of users. This happens for several reasons:

\begin{itemize}
\item The fraction of time according to which cloudlet is is busy increases, in other words the period that cloudlet is non-idle is longer; therefore, since more time is needed to complete user tasks, more power are consumed.

\item aumenta lamda
\end{itemize} 

is the processing time is decreased increasing the number 



\end{enumerate}





\subsection{Second experiment set}

\citet{MSAReport} perform a second experiment set in order to study the benefits of using edge computing resources, that is study the impact of installing cloudlet on WBAN system performance. 
To be more precise, that simulation set is focus on monitoring the effects on average transmission power and delay of data packed send by users PDA to cloudlet due to edge resources, varying the number of user, deployed cloudlet and their geographical placement, while monitored area size is fixed to 600 x 400 m.

Here are the results of the simulation:

\begin{enumerate}

\item Increasing the number of available cloudlet in monitored area, average transmission delay of data packed is reduced. There are a number of reasons that explain this phenomenon:

\begin{enumerate}

\item Increasing the number of deployed cloudlet, the probability that an user is close to one of them increase too. It is shown that cloudlet \textbf{physical proximity} to a mobile device \textit{can} decrease RTT, therefore affecting positively on latency, bandwidth and jitter. \cite{TheEmergenceOfEdgeComputing}. 

However, is very important to precise that generally cloudlet physical proximity not always can affect positively RTT. In fact, data packet transmission performances depends on the so-called \textbf{network proximity}, according to which RTT is low and end-to-end bandwidth is high. This is achievable by using a fiber link between a wireless access point and a cloudlet that is many tens or even hundreds of kilometres away. Conversely, physical proximity does not guarantee network proximity since a highly congested WiFi network may have poor RTT.\cite{TheSeminalRoleEdgeNativeApplications}

\item When an WBAN users can transmit data packet within cloudlet coverage, average transmission delay is affected by the nature of connection too. 

In fact, when cloudlet coverage isn't available, data is offloaded to cloud using a \textbf{multi-hop network connections} and moreover, although bandwidth between the mobile device and the first hop (i.e., the first mile) and the last hop and the cloud resource (i.e., the last mile) is potentially high bandwidth, the bandwidth between the rest of the hops is likely low. Therefore RTT is very high.\cite{ArchitecturalTacticsCyberForaging}

Conversely, when cloudlet coverage is available, data is offloaded to a cloudlet using a high bandwidth single-hop connection, therefore RTT decrease.\cite{ArchitecturalTacticsCyberForaging}

\item Increasing the number of available cloudlet in monitored area, average transmission power of data packed is reduced. It is due to the use of WiFi technology instead to cellular network

\item Increasing the number users, both average transmission power and delay increase too. 

Since a polling MAC scheme is used, an higher number of users within an cloudlet WiFi coverage area increasing the time needed to pool every node by AP, in other words the so-called contention-free period is longer. 

According to \citep{MSAReport}, an high congested Wifi network can cause WBAN devices to perform a self-adtaption beahivour accoding to which they switch comunication technoogy, using cellular network to send packet data.




\end{enumerate}

\end{enumerate}


\subsection{Third experiment set}

The last experiment set is focusing on monitoring aforementioned performance metric varying cloudlet placement and, precisely, \textbf{inter-cloudlet distance}. For a given monitored area, which size is now fixed to $800 \times 800 \; m$, \citet{MSAReport} had classified cloudlet deployment into three categories:

\begin{description}
\item[Adjacent deployment] minimum inter-cloudlet distance.
\item[Distant deployment] maximum inter-cloudlet distance.
\item[Distant deployment] medium inter-cloudlet distance.
\end{description}

For each cloudlet deployment category, \citet{MSAReport} had performed several simulations varying number of users, of virtual machines and of cloudlet. Experiments results are the following:

\begin{enumerate}

\item Independently from cloudlet deployment category, increasing the number of cloudlet, the impact of inter-cloudlet distance on average transmission power and delay is negligible. It is true because, increase the opportunities of the users to send the data packet to the corresponding VC using WiFi and with minimum cost of power and delay, rather than of using cellular connection.

\item Intermediate Deployment performs better since show lower average transmission power and delay compared with other deployments, while Adjacent deployment performs worse. As said prevouslty, difdferences about cloudlet deployiment tend to vanish incresing the number of cloudlet.  

\end{enumerate}




















\section{MAC protocol}

According to \citet{MSAReport}, the medium access mechanisms used in their simulation is \textbf{pooling-based}, that is a strictly centralized scheme, also called \textbf{point coordination function} (\textbf{PCF}), in which the access point, called \textbf{master}, dynamically polls clients for data. According to this scheme, the master is allowed to build a list of stations wishing to transmit during a contention phase. After this phase, the station polls each station on the list.

In a WBAN's context, this choice has several advantages:

\begin{itemize}

\item PCF is capable to offer time-bounded service, \textbf{guaranteeing a maximum access delay and minimum transmission bandwidth} making the system more \textbf{predictable} and, therefore, more suitable \textbf{real-time} health monitoring; in fact, if a predictable system is used, will be possible to know with reasonable approximation the \textit{worst case execution time} (WCET) of jobs and check if any hard real-time constraints is respected.

\item Since WBAN users are not allowed to send data without the master's invitation, the \textbf{hidden terminal problem is eliminated}. Keep in mind that using DCF (Distributed coordination function) with RTS/CTS extension, in order to resolve hidden terminal problem, is not suitable in a WBAN's context because to short frame (only 46 byte) send by devices; RTS/CTS extension would introduce a non-negligible overhead causing a waste of bandwidth, higher delay and energy consumption.

\item This scheme allow higher throughput due to less collisions respect to CSMA/CA.

\end{itemize}

Obliviously this choice isn't a panacea; it can introduce higher delay under a light load and overhead if nodes have nothing to send.

\subsection{System Architecture}

Edge Computing 

It tends to lengthen
network round-trip times (RTT) from mobile users, and to
increase cumulative ingress bandwidth demand from Inter-
net of Things (IoT) devices.

Remote: Computation or data is offloaded to a remote resource
such as an enterprise cloud or data center, as shown in part (a)
of Fig. 3. In this case, it is a synchronous operation that requires
multiple network hops between the mobile device and the enter-
prise cloud. Even though the bandwidth between the mobile de-
vice and the first hop (i.e., the first mile) and the last hop and the
cloud resource (i.e., the last mile) is potentially high bandwidth,
thebandwidthbetweentherestofthehopsislikely low.


\citet{MSAReport} 












\section{Minimizing \textit{Packet Transmission Power}}


It was shown that, via WiFi, the transmission power of a data packet of size 46 Bytes will cost about
30 mw [29,31,32] and with a delay of 0.045 ms. On the other hand, a longer transmission range cellular network connection
(e.g. 3G and LTE) is capable of transmitting the data packet to the cloud from any location that is cover by cellular network,

which is usually a wider geographic area compared with the WiFi. It was shown that, via cellular, the transmission power of
data packet of size 46 Bytes will cost about 300 mw and with a delay of 0.45 ms

\section{}

In this section, we will describe other techniques and designs capable to reduce power consumption in WBAN devices.

\subsection{Approximate Computing Techniques}




\subsection{RRRRRRRRRR}

\begin{quoting}[font=itshape, begintext={``}, endtext={''\cite[par.~3.1]{MSAReport}}]
It was shown that, via WiFi, the transmission power of a data packet of size 46 Bytes will cost about 30 mw and with a delay of 0.045 ms. On the other hand, a longer transmission range cellular network connection (e.g. 3G and LTE) is capable of transmitting the data packet to the cloud from any location that is cover by cellular network, which is usually a wider geographic area compared with the WiFi. It was shown that, via cellular, the transmission power of data packet of size 46 Bytes will cost about 300 mw and with a delay of 0.45 ms [...]


With WiFi, a WBAN user will be able to transmit the data packet to the cloud with low power and low delay compared with cellular technology, but with transmission range does not exceed 100 m. [...]


While the cellular connection is very costly in terms of power, transmission delay and connection cost (the WiFi is mostly free of charge),
\end{quoting}

\vspace{0.3cm}

According to \citet{MSAReport}, including many other researches cited by them, 

Is known that energy is one of the most valuable resources in a WBAN.

According to researchers of this study, the key method to reduce power consumption (and connection cost) is to develop a \textbf{self-adaptive system capable to change data communication technology dynamically when users move from one region to another}. 

To be more precise, if WiFi coverage is available, users will use WiFi technology to transmit a data packet, otherwise an user can use only cellular technology. If neither WiFi or cellular technology is available, users will buffer the packets until one of the above technologies is available.\cite[par.~3.1]{MSAReport}


\subsection{}

Unfortunately, researchers don't specify which version of the standard IEEE 802.11 is used in their simulations, however they declare that transmission range of WiFi base stations does not exceed 100 m. 

We believe that this system model can hardly be used for offloading in outdoor scenarios owing to short transmission range of WiFi stations and high economical effort to build proposed infrastructure.



In order to perform their simulations and study the impacts of cloudlet servers on packet delay and power consumptions, researchers model a large scale WBANs system using two monitored areas, one of $600$ x $400$ $m$ and $800$ x $800$ $m$, in which several users are moved in randomly. \citet{MSAReport} perform several simulations varying also the number of users and cloudlet.





 800 800 m = 

Researchers declare to study the impacts of installing cloudlet servers on the data collection performance in the monitoring area, the
following experiments were carried out. 400 users were moved in a random waypoint model within a monitored area of
600 400 m.


Although 802.11n and 802.11.ac data rates are comparable to or
even higher than the rates of the modern mobile networks, e.g. LTE, they


a 100 m transimmission range







The 802.11ah standard enables single-hop communication over distances up to 1000 m.



However, it is not suited for facilitating commu-
nication among a large number of IoT devices or for covering
large areas.











The power consumption of wireless transceivers is 
substantially impacted by their transmission power, which 
can be responsible for up to 70% of the total power 
consumption for off-the-shelf sensor nodes, e.g.

Energy is one of the most valuable resources in a 
WBAN. One of the major sources of energy waste in 
WBANs originates from internetwork interference between 
nearby WBANs which causes the Signal-to-Interference-
and-Noise Ratio (SINR) to drop and thereby throughput 
degrades.


----------

Roma - 1287,36 km2 = 1287360000 m2

2.011,5 cloudlet  


3.240.000




Medium Access
Control (MAC) protocols play a significant role in maximizing
the lifetime of WBAN by controlling the dominant sources
of energy waste, i.e., collision, idle listening, overhearing,
and control packet overhead.

To achieve communication over longer distances among a large number of low-power devices, several innovative concepts 

The 802.11ah standard enables single-hop communication over distances up to $1000 m$.


As expected, \citet{MSAReport} results demonstrate that increasing the number of users, for a given cloudlet deployment, will increase the power and delay of the collected data; this happens due an higher probability of collision and to interference between nearby WBAN devices, which causes the Signal-to-Interference-and-Noise Ratio (SINR) to drop and thereby throughput degrades.


 The
default medium access control (MAC) protocol for channel access in this standard is the distributed coordination
function (DCF), which is a random access scheme based
on carrier sense multiple access with collision avoidance
(CSMA/CA).

In addition to the basic access, an optional four way handshaking technique, known as request-to-send/clear-to-send
(RTS/CTS) mechanism has been standardized. Before transmitting a packet, a station operating in RTS/CTS mode “reserves”
the channel by sending a special Request-To-Send short frame.
The destination station acknowledges the receipt of an RTS
frame by sending back a Clear-To-Send frame, after which
normal packet transmission and ACK response occurs. Since
collision may occur only on the RTS frame, and it is detected
by the lack of CTS response, the RTS/CTS mechanism allows
to increase the system performance by reducing the duration
of a collision when long messages are transmitted. As an
important side effect, the RTS/CTS scheme designed in the
802.11 protocol is suited to combat the so-called problem of
Hidden Terminals


 The primary medium access control (MAC) technique of 802.11 is called distributed coordination function (DCF). DCF is a carrier sense multiple access
with collision avoidance (CSMA/CA) scheme with binary slotted
exponential backoff




the overheads associated with
the RTS and CTS transmissions – for short and time-critical data packets, this is not negligible 


\section{Medium Access Control}

The power consumption of wireless transceivers is substantially impacted by their transmission power, which can be responsible for up to \textbf{70\%} of the total power consumption. It was shown that the use of a Medium Access Control (MAC) protocols capable to minimize the dominant sources of energy waste like \textit{collision}, \textit{idle listening} and \textit{control packet overhead} is extremely useful to maximizing the lifetime of WBAN devices.

Unfortunately, \citet{MSAReport} don't give any details about used MAC protocol to transmit aggregated data packet to cloudlet, not even specifying the version of the standard IEEE 802.11 used in their simulations, although they declare that its transmission range does not exceed 100 m. Therefore, for our analysis we assume that \citet{MSAReport} have used legacy IEEE 802.11 protocol with default medium access control protocol for channel access, that is the distributed coordination function (DCF), which is a random access scheme based on carrier sense multiple access with collision avoidance (CSMA/CA).

In this section we try to prove that the use of traditional legacy IEEE 802.11 protocol can hardly be used for offloading in outdoor scenarios.

To achieve communication over longer distances among a large number of low-power devices, we can exploit the several innovative concepts introduced in the PHYsical (PHY) and Medium Access Control (MAC) layers of 802.11ah standard. As known, the 802.11ah standard enables single-hop communication over distances up to $1000 m$ and, thanks to Unlicensed < 1GHz band, better propagation characteristics in outdoor scenarios than traditional for Wi-Fi 2.4 and 5 GHz bands.

The Carrier Sense Multiple Access with Collision Avoid-ance (CSMA/CA)-based MAC of 802.11 fails to handle collisions, especially when thousands of devices simultaneously contend for a channel. To address this issue, the 802.11ah introduces a Restricted Access Window (RAW)-based mechanism that improves the network performance significantly. This mechanism splits the STAs into groups and allows the STAs belonging to a certain group to access the medium at any particular time frame.

Although .11ah PHY provides high data rates, the aggregate throughput may be very low because of the huge overhead inherent to short packet transmission typical for sensor networks.

o reduce it, the draft standard defines a new backward incompatible
format of shortened headers for data, management and novel control frames








\section{gd}

Re










nother approach is to reduce the
fidelity of the underlying algorithms when edge computing is
not available.


\section{The cloudlet discovery issue}

Is obvious that WBAN users need to be able to locate available cloudlet in an area on which stage data, but this issue has not been addressed by \citet{MSAReport} in any way; we believe that it is a very big mistake because it was shown that cloudlet discovery affects energy consumptions significantly and, more importantly, it is a functional requirements for any cyber-foraging systems.

In WBAN systems, maximize energy efficiency,in order to preserve devices lifetime is critical but, since transmitted data are very sensitive, we believe that security is very important too. Therefore, a possible solution to cloudlet discovery issue is the use of \textbf{Cloud Surrogate Directory} tactic, according to which the mobile device contacts a cloud server that maintains a list of potential cloudlet/surrogates. 

Using this tactic Security is highly increased because the mobile device only needs to trust the cloud surrogate directory server and can pre-exchange credentials for authorization. The surrogate directory server can also exchange credentials with its surrogates as part of the registration process, which means that the directory would only contains trusted surrogates.

Because the selection algorithm of optimal cloudlet runs in the cloud it also increases energy efficiency. 

However, response/execution time can be increased because of the the additional directory
query time. In addition, availability is negatively affected
because the mobile device requires consistent connectivity to
the cloud at least in the discovery phase, which means that
the cloud server becomes a single-point-of-failure.

\section{Mobility management issue}

\section{Economical issues}

The lower the frequency, the better the penetration. Long waves can be transmitted through the oceans to a submarine while high frequencies can be blocked by a tree. The higher the frequency, the more the behaviour of the radio waves resemble that of light.

As known, the 802.11ah standard enables communication over distances up to $1000 m$ and, since it use a lower frequency (< 1GHz) it has better propagation characteristics in outdoor scenarios than traditional for Wi-Fi 2.4 and 5 GHz bands. 



\section{dsadas}

\citet{MSAReport} report lacks of a specification model capable to turn needed system variables 
’s states into a collection of mathematical variables together with equations and logic describing how the state variables are interrelated, including algorithms for computing their interaction and evolution in time

We believe that the study 

For example, \citet{MSAReport} doesn't report how \textit{users movements speed} (fixed to $2 m/s$) and the \textit{random pause time} (fixed to $1-10 s$) affected simulation's results or how they are intercorrelated with average power consumption and delay of transmitting.

Is not clear how they have modeled stochastic components of system, like packet arrival process into the cloudlet or their service time.
are they xponentially distributed, Uniformly distributed?


\subsection{Approximate computing techniques}

Everyday experience teaches us that battery technology is the very bottleneck of every mobile devices, WBAN devices included, and, unfortunately, as known battery technology is advancing very slowly if compared to the others computer technology, like CPU, storage systems and RAM.

A promising solution is \textbf{approximate computing} (and \textbf{storage}), which is based on the intuitive observation that while performing exact computation require high amount of resources, allowing selective approximation or imperfect computations results can provide drastic energy savings.

Today many important mobile applications use a some kind of approximation computing technique, including computer vision, sensor data analysis, machine learning, augmented reality and signal processing. 

Approximate computing leverage the presence of error-tolerant code regions in applications and per-
ceptual limitations of users to intelligently trade off implementation, storage and/or
result accuracy for performance or energy gains. In brief, AC exploits the gap between
the level of accuracy required by the applications/users and that provided by the com-
puting system, for achieving diverse optimizations.





\bibliographystyle{ACM-Reference-Format}
\bibliography{Bibliography}

\appendix

\end{document}
\endinput
