\documentclass[sigchi]{acmart}
\usepackage{algorithm}
\usepackage[noend]{algpseudocode}
\usepackage{natbib}
\usepackage{quoting}

\AtBeginDocument{%
  \providecommand\BibTeX{{%
    \normalfont B\kern-0.5em{\scshape i\kern-0.25em b}\kern-0.8em\TeX}}}

\settopmatter{printacmref=false, printccs=false, printfolios =false}
\setcopyright{none} 
\renewcommand\footnotetextcopyrightpermission[1]{}

\acmConference[MSA A.A. 2019-2020]{ }{September, 2020}{ }

\begin{document}

\title{Analysis of "Cloudlet-based Efficient Data Collection in Wireless Body Area Networks" report}

\author{Andrea Graziani}
\email{andrea.graziani93@outlook.it}
\affiliation{%
  \institution{Università Degli Studi di Roma Tor Vergata}
  \city{Rome}
  \state{Italy}
}

\renewcommand{\shortauthors}{Andrea Graziani (0273395)}

\maketitle

\section{Introduction}

\subsection{Research's Goal}

\vspace{0.3cm}

\begin{quoting}[font=itshape, begintext={``}, endtext={''\cite[par.~1.4]{MSAReport}}]
The main goal of this paper is to develop a large scale WBANs system in the presence of cloudlet-based data collection model. The objective is to minimize end-to-end packet cost by dynamically choosing data collection to the cloud by using cloudlet based system [...] reducing packet-to-cloud energy, the proposed work also attempt to minimize the end-to-end packet delay
\end{quoting}

\vspace{0.3cm}

According to \citet{MSAReport}, the goal of their work is to built an efficient \textbf{Wireless Body Area Networks} (\textbf{WBAN}) exploiting \textbf{edge computing}, a new paradigm in which substantial computing and storage resources, referred to as \textbf{cloudlets}, are placed at the Internet's edge, that is in close proximity to WBAN devices or sensors.\cite{TheEmergenceOfEdgeComputing}

To be more precise, as stated by \citet{MSAReport}, edge computing resources are exploited in order to \textbf{minimize average power consumption and delay} when transmitting collected data by \textbf{Personal Digital Assistant} (\textbf{PDA}) device to cloud. 

\subsection{Edge Computing Application Typology}

\vspace{0.3cm}

\begin{quoting}[font=itshape, begintext={``}, endtext={''\cite[par.~3.2]{MSAReport}}]
In our implementation, the enterprise cloud system will be the ultimate destination of the collected data. The enterprise cloud is also able to send messages back to both the cloudlet system or to WBAN users.
\end{quoting}

\vspace{0.3cm}

\begin{quoting}[font=itshape, begintext={``}, endtext={''\cite[par.~4.1]{MSAReport}}]
The enterprise cloud system is a centralized management and storage point that can be accessed by different organizations that are interested in a certain type of data. Another important feature of the cloudlet system is the ability of bidirectional communications between many WBANs users.
\end{quoting}

\vspace{0.3cm}

WBAN devices and sensors have very stringent constraints in term of CPU performance, storage resources and battery life to perform a wide range of healthcare, military and sport application. This is the reason because \textbf{Tier-1}, which represents "\textit{the cloud}" in today's parlance, plays a very important role in this kind of system: cloud resources are exploited to overcome WBAN devices penalty, since Tier-1 represents, in terms of archival preservation, the \textit{safest place to store data}, ensuring the long-term integrity and accessibility, and the \textit{main execution site to perform expensive computations}, thanks to its almost unlimited elasticity. 

However cloud resources exploitation isn't a panacea because there are some negative consequence, as the increasing of network \textbf{round-trip times} (\textbf{RTT}) experienced by mobile users; therefore, \citet{MSAReport} solution is to built a system capable to exploit edge computing resources, which, as it has shown, allow to offload compute-intensive operations at very low latency.\cite{TheSeminalRoleEdgeNativeApplications}\cite{TheEmergenceOfEdgeComputing}

To be more precise, WBAN system prototype modelled by \citet{MSAReport} is intended for \textit{edge-accelerated, cloud-native applications}. Why?

\begin{description}

\item[cloud-native] because, to overcome stringent constraints on battery, performance and storage of WBAN devices, cloud resources are exploited, through offloading techniques over a wireless network, \textbf{to execute critical task}; in other words, as stated also by \citet{MSAReport}, \textit{cloud is the primary execution site and it is essential to provide services to final users}.

\item[edge-accelerated] because the system exploits edge computing resources \textbf{only when available}. In that way, that system is capable to provide optimal performance, but \textit{edge resources represents a secondary execution site, that is they are optional to provide services}.

\end{description}

According to \citet{TheSeminalRoleEdgeNativeApplications}, it is shown that edge-accelerated, cloud-native applications can have a \textit{modest} advantages in term of \textit{response times}, \textit{battery life}, \textit{ingress bandwidth demand reduction}, \textit{privacy} and \textit{fallback services}\cite{TheSeminalRoleEdgeNativeApplications}\citep{TheEmergenceOfEdgeComputing}, but, as \citet{TheSeminalRoleEdgeNativeApplications} states, edge computing potentiality is not fully exploited due to central role assigned to cloud.

However, it is true that any edge-accelerated, cloud-native applications involves less investment risk, much less software development, and their markets are much larger since they \textit{can function acceptably even in the absence of edge computing}.

\subsection{Edge Computing Resources exploitation}

How edge computing resources are exploited in order to improve, as stated by \citet{MSAReport}, average power consumption and delay during packet transmissions? 

\subsubsection{Using an energy-efficient communication technology}

First of all, is very important to precise that \citet{MSAReport} research is based on the observation that using WiFi technology a WBAN user will be able to transmit data packet to the cloud with \textbf{low power consumption}, \textbf{low delay} and mostly \textbf{no connection cost} compared with cellular technology; in fact, they wrote:

\vspace{0.3cm}

\begin{quoting}[font=itshape, begintext={``}, endtext={''\cite[par.~3.1]{MSAReport}}]
It was shown that, via WiFi, the transmission power of a data packet of size 46 Bytes will cost about 30 mw and with a delay of 0.045 ms. On the other hand, a longer transmission range cellular network connection (e.g. 3G and LTE) is capable of transmitting the data packet to the cloud from any location that is cover by cellular network, which is usually a wider geographic area compared with the WiFi. It was shown that, via cellular, the transmission power of data packet of size 46 Bytes will cost about 300 mw and with a delay of 0.45 ms.
\end{quoting}

\vspace{0.3cm}

Therefore, since WiFi technology is more energy-efficient than cellular technology, if WBAN users have the chance to use it, is possible to reduce average power consumption during packet transmission. 

However, how to exploit WiFi technology in WBAN system taking into account its \textbf{short transmission range} compared to cellular network? Providing an edge computing infrastructure made up of several cloudlets geographically distributed and equipped with computational, storage and communication capability, increasing WiFi coverage. In fact, researchers wrote:

\vspace{0.3cm}

\begin{quoting}[font=itshape, begintext={``}, endtext={''\cite[par.~4.1]{MSAReport}}]
The cloudlet system is composed of set of physical servers with many cores and huge Gigabytes of memory. The cloudlet server system is equipped with one or more of the communication antennas that is supporting different physical layer capabilities (e.g. WiFi and WiMax). 
\end{quoting}

\vspace{0.3cm}

However there is a problem: since WiFi technology transmission range is short and edge infrastructure resources are expensive,  cloudlet and WiFi coverage aren't available everywhere. Therefore, \citet{MSAReport} implemented a \textit{self-adaptive behaviour} according to which \textbf{communication technology is dynamically changed} when users move from one region to another. This is the reason according to which \citet{MSAReport} modelled their system identifying three different regions:\cite[par.~3.1]{MSAReport} 

\begin{description}

\item[Cloudlet Region (CR)] where WiFi coverage is available, so a user can use it to transmit data to the cloudlet.

\item[Enterprise Region (ER)] where only cellular coverage is available, therefore  a user can use only cellular technology to transmit a data packet to cloud. 

\item[Not-covered Region (NC)] where neither WiFi nor cellular technology is available. In this case a user should buffer the packets until one of the above technologies is available, then to be able of transmitting the packet to the enterprise cloud. 

\end{description}

In other words, WBAN users are capable to exploit edge computing resources if and only if they are in a \textit{Cloudlet Region}, otherwise they go on transmitting data directly to the cloud, if possible. This is the reason according to which we have previously defined \citet{MSAReport} application as \textit{edge-accelerated}: cloud is very critical for that application since it represents the ultimate destination of collected data and perform critical task providing access to data. In other words, seem enough clear that cloudlet is less critical compared to cloud to providing system services. In fact, \citet{MSAReport} wrote:

\vspace{0.3cm}

\begin{quoting}[font=itshape, begintext={``}, endtext={''\cite[par.~3.2]{MSAReport}}]
In our implementation, the enterprise cloud system will be the ultimate destination of the collected data. The enterprise cloud is also able to send messages back to both the cloudlet system or to WBAN users.
\end{quoting}

\vspace{0.3cm}

\begin{quoting}[font=itshape, begintext={``}, endtext={''\cite[par.~4.1]{MSAReport}}] 
The enterprise cloud system is a centralized management and storage point that can be accessed by different organizations that are interested in a certain type of data. 
\end{quoting}

\vspace{0.3cm}

\subsubsection{Exploiting network proximity}

Deploying cloudlet \textit{close} to final users, thanks to \textbf{high-bandwidth single-hop connection}, is possible to reduce RTT and to increase end-to-end bandwidth; in other words, we can achieve network-proximity.\citep{TheSeminalRoleEdgeNativeApplications}

However, is very important to not confuse \textit{physical proximity} with \textit{network-proximity}. Cloudlet physical proximity not always can affect positively RTT since it does \textit{not} guarantee network proximity: think for example to a highly congested WiFi network which may have poor RTT. Theoretically, is possible to achieve network proximity without physical proximity, for example using a fiber link between a wireless access point and a cloudlet that is many tens or even hundreds of kilometres away assuring therefore low RTT and high bandwidth.\cite{TheSeminalRoleEdgeNativeApplications}

\subsubsection{Using cyber-foraging techniques}

WBAN sensors and devices collect a very huge amount of data. In fact, \citet{MSAReport} wrote:

\vspace{0.3cm}

\begin{quoting}[font=itshape, begintext={``}, endtext={''\cite[par.~1.1]{MSAReport}}]
The multiple WBAN sensor nodes are capable of sampling, processing, and communicating one or more vital signs like heart rate, blood pressure, oxygen saturation, breathing rate, diabetes, body temperature, ECG and activity, or environmental parameters like location, temperature, humidity, light, movement, proximity and direction. 
\end{quoting}

\vspace{0.3cm}

Obliviously, WBAN devices and sensors cannot have enough computation and storage resources due to their very strictly constrains in term of size, heat and power consumption. In order to overcome to device limitation, increasing energy efficiency.

In order to overcome to these limitations, \citet{MSAReport} adopt a \textbf{Data Staging Tactic} to allow WBAN devices, after offloading
collected data to cloudlet or cloud, to free up storage space. To achieve energy-efficiency, all computations regarding collected data 
are carried out by cloud or cloudlet, therefore \textbf{Computation Offload Tactic} is used too.


\section{Simulations analysis}

In these section we will analyse simulation's results performed by \citet{MSAReport} in order to quantify performance advantages by using edge computing resources. 

WBAN system performance is evaluated through several simulation sets, focusing on two performance metrics:

\vspace{0.3cm}

\begin{quoting}[font=itshape, begintext={``}, endtext={''\cite[par.~4.2]{MSAReport}}]
\textit{Packet Transmission Power} and \textit{Packet Delay}. The Packet Transmission Power and Packet Delay are directly measure of the communication energy and delay expenditure from the Personal Digital Assistant (PDA) device to the cloudlet or the enterprise cloud
\end{quoting}

\vspace{0.3cm}

As stated by \citet{MSAReport}, experiments were carried out using a custom version of \textit{CloudSim}\footnote{\url{http://www.cloudbus.org/cloudsim/}} simulator, monitoring a virtual space-area, where several cloudlet entities were deployed in known geographic locations, in such a way transmission range belonging to different cloudlet not overlapping, while some WBAN users were moved randomly with fixed speed and random pause time.

\subsubsection{MAC protocol}

According to \citet{MSAReport}, the medium access mechanisms used in their simulation is \textbf{pooling-based}, that is a strictly centralized scheme, also called \textbf{point coordination function} (\textbf{PCF}), in which the access point, called \textbf{master}, dynamically polls clients for data. According to this scheme, the master is allowed to build a list of stations wishing to transmit during a contention phase. After this phase, the station polls each station on the list.

In a WBAN's context, this choice has several advantages:

\begin{itemize}

\item PCF is capable to offer time-bounded service, \textbf{guaranteeing a maximum access delay and minimum transmission bandwidth} making the system more \textbf{predictable} and, therefore, more suitable \textbf{real-time} health monitoring; in fact, if a predictable system is used, will be possible to know with reasonable approximation the \textit{worst case execution time} (WCET) of jobs and check if any hard real-time constraints is respected.

\item Since WBAN users are not allowed to send data without the master's invitation, the \textbf{hidden terminal problem is eliminated}. Keep in mind that using DCF (Distributed coordination function) with RTS/CTS extension, in order to resolve hidden terminal problem, is not suitable in a WBAN's context because to short frame (only 46 byte) send by devices; RTS/CTS extension would introduce a non-negligible overhead causing a waste of bandwidth, higher delay and energy consumption.

\item This scheme allow higher throughput due to less collisions respect to CSMA/CA.

\end{itemize}

Obliviously this choice isn't a panacea; it can introduce higher delay under a light load and overhead if nodes have nothing to send.

\subsection{First experiment set}

First experiment set goal is to quantify the impact on \textbf{packet process delay}, also known as \textit{average service time}, and on \textbf{power consumption} due to packet computation by cloudlet by varying following system parameter.

\begin{enumerate}
\item number of virtual machine deployed on a cloudlet (from 0 to 8).
\item processing speed of data packet, from a minimum 100 to a maximum of 900 \textbf{million instructions per second} (MIPS).
\item number of WBAN users (up to 150 users).
\end{enumerate}

According to experiments results, \cite{MSAReport} have shown that:

\begin{enumerate}

\item Increasing the number of virtual machine running on cloudlet, assigning each one a processor, system cloudlet \textbf{utilization}, that is \textit{the fraction of time according to which cloudlet is busy increases}, was been reduced, decreasing therefore processing time.

In fact, fixed data packet average arrival rate $\lambda$ and average service rate of each virtual machine $\mu$, each virtual machine, by symmetry, sees an arrival rate of $\dfrac{\lambda}{k}$, where $k$ is the number of deployed virtual machines. Hence the cloudlet utilization is $\dfrac{\lambda}{k\mu}$, which is less than $\dfrac{\lambda}{\mu}$ using only one server.  

Remember that higher utilization involves an higher average number of data packet to process, increasing waiting time experienced by final user.

\vspace{0.3cm}

\begin{quoting}[font=itshape, begintext={``}, endtext={''\cite[par.~3.3]{MSAReport}}]
the processing time is decreased by approximately 85\% by using cloudlet system configured with 8 VMs comparing to using only one VM in CS.
\end{quoting}

\vspace{0.3cm}

\item Fixed the average time required to process a packet data on a CPU, that is for a given MIPS process speed, power consumption and processing delay are increased by increasing the number of users. This happens because cloudlet utilization increases since $\lambda$ is higher, therefore the fraction of time according to which cloudlet is busy increases; so, since more time is needed to complete user tasks, more power are consumed.

\end{enumerate}

\subsection{Second experiment set}

Second experiment set was carried out in order to quantify advantages of using edge computing resources. To be more precise, that simulation monitors the effects on \textbf{average transmission power and delay} of data packed send by users PDA to cloudlet, varying following system parameters:

\begin{enumerate}
\item number of cloudlet deployed (from 0 to 6).
\item WBAN user's positions (speed of $2\;m/s$ and a random pause time of $1-10\;s$)
\end{enumerate}

Monitored area size is fixed to $600 \times 400\;m$ while the total number of users is $400$.

According to experiments results, \cite{MSAReport} have shown that:

\begin{enumerate}

\item As expected, increasing the number of available cloudlets in monitored area, average transmission delay of data packed is reduced. It happens since the probability  that an user is close to one of them increases. Having more opportunities to transmit data within cloudlet coverage, users can benefit of cloudlet physical proximity which, as already said in previous section, \textit{can} decrease RTT, affecting positively latency, bandwidth and jitter. \cite{TheEmergenceOfEdgeComputing}. 

Moreover, when cloudlet coverage isn't available, data is offloaded to cloud using a \textbf{multi-hop network connections} which involves high RTT and likely low inter-hop bandwidth, although bandwidth between the mobile device and the first hop (i.e., the first mile) and the last hop and the cloud resource (i.e., the last mile) is potentially high bandwidth.\cite{ArchitecturalTacticsCyberForaging}

Conversely, when cloudlet coverage is available, data is offloaded to a cloudlet using a high bandwidth single-hop connection, which decrease RTT.

\item Increasing the number of available cloudlets, average transmission power of data packed is reduced too. It is due to the use of WiFi technology which is, as already said, more energy-efficient compared to cellular network.

\item Increasing the number of users, both average transmission power and delay increase. This happens mainly due to high congested Wifi network; since a polling MAC scheme is used, an higher number of users within an cloudlet WiFi coverage area increases the time needed to pool every node by AP. On the other hand, as stated by \citep{MSAReport}, an high congested Wifi network can cause WBAN devices to perform a self-adaptation action according to which communication technology is switched, using cellular network to send packet data instead WiFi.

\end{enumerate}

\subsection{Third experiment set}

The last experiment set is focusing on monitoring aforementioned performance metric varying cloudlet geographical placement. Monitoring an $800 \times 800 \; m$ area, simulations were carried out varying following system parameters:

\begin{enumerate}
\item number of cloudlet deployed (up to 16 cloudlets).
\item number of WBAN users (up to 1400 users).
\item WBAN user's positions (same parameters as before)
\item cloudlet geographical placement (using very different patterns, classified in three categories by \citet{MSAReport}: \textit{Adjacent}, \textit{Distant} and \textit{Intermediate})
\end{enumerate}

Experiments results are the following:

\begin{enumerate}

\item As expected, independently from cloudlet deployment pattern, increasing the number of cloudlet, the impact of cloudlet geographical placement on average transmission power and delay is negligible since, in that way, the opportunities to send the data packet to a cloudlet, using WiFi and with minimum cost of power and delay, increase.

\item Fixed cloudlet and users number, deploying cloudlet using an intermediate category pattern, that is placing cloudlet neither too far apart nor too close, system performance are better than other patterns belonging to other categories.

\end{enumerate}

\section{Issues}

In this section, we will presents some issues that, we believe, reduce \citep{MSAReport} study quality.

\subsection{The cloudlet discovery and provisioning issue}

As said in previous section, \citet{MSAReport} overcome to WBAN devices limits, in term of computing power and storage, through cyber-foraging techniques like computation offload and data staging. We believe that the system model built by \citet{MSAReport} is too simple to achieve research's goals.

It is shown \cite{DecisionModel} that cyber-foraging systems have \textit{at a minimum} the following combination of functional requirements:

\begin{itemize}

\item  A need for computation offload, data staging, or both
\item  A need to provision a surrogate with the offloaded computation or data staging capabilities
\item  A need for the mobile device to locate a surrogate at runtime

\end{itemize}

Although WBAN users \textit{need} to be able to locate available cloudlet in an area where stage data, \textbf{cloudlet discovery} issue has \textit{not} been addressed by \citet{MSAReport} in any way; we believe that it is a very big mistake because cloudlet discovery affects negatively energy consumptions and response time and, anyway, it is a functional requirements for any cyber-foraging systems. 

However there is another reason which that makes cloudlet discovery issue so important: \textbf{security}. \citep{MSAReport} wrote:

\vspace{0.3cm}

\begin{quoting}[font=itshape, begintext={``}, endtext={''\cite[par.~3.3]{MSAReport}}]
On the other hand, since WBANs forward useful and life-critical information to the cloud, which may operate in distributed and hostile environments, novel security mechanisms are required to prevent malicious interactions to the storage infrastructure. Both the cloud providers and the users must take strong security measures to protect the storage infrastructure.
\end{quoting}

\vspace{0.3cm}

\citet{MSAReport} should have known that addressing cloudlet discovery issue, they could have improved system security. For example, using \textbf{Cloud Surrogate Directory} tactic, according to which mobile device contacts a cloud server that maintains a list of potential cloudlet/surrogates, security is highly increased because the mobile device only needs to trust the cloud surrogate directory server and can pre-exchange credentials for authorization. 

Generally, when you want to offload data or computation to a cloudlet, a selection algorithm to select best cloudlet is run. In WBAN systems, maximize energy efficiency, in order to preserve devices lifetime, is critical, therefore is preferable not run selection algorithm on WBAN devices, which can decrease energy efficiency (depending on the complexity of the algorithm and the number of monitored variables). Probably, using cloud surrogate directory tactic, according to which selection algorithms run on the cloud, is possible to improve WBAN lifetime. \cite{DecisionModel}

In the same way, \textbf{surrogate provisioning} is another functional requirements for any cyber-foraging systems which is not addressed by \citep{MSAReport}, therefore is not clear how cloudlets manage offloaded computation and/or data processing operations.

During the first experiment set, \citet{MSAReport} had shown that packet process delay is affected by the number of virtual machine deployed in cloudlet. However, provisioning tactic affects this process delay too. It is shown that Pre-provisioned cloudlet have the advantage of shorter provisioning times because the capabilities already reside on cloudlet, providing shorter response times to requests from mobile devices.\cite{DecisionModel}

\subsection{The lack of a specification model}

We believe that \citet{MSAReport} study is hardly reproducible due to the lack of an adequate \textbf{specification model} capable to highlight system state variable and equations and logic describing how the state variables are interrelated, including algorithms for computing their interaction and evolution in time.

For example, \citet{MSAReport} doesn't report how \textit{users movements speed} (fixed to $2 m/s$) and the \textit{random pause time} (fixed to $1-10 s$) affected simulation's results or how they are intercorrelated with average power consumption and delay during the second and the third experiment set.

Cloudlet deployment category 

There is no hint about scheduling policy . Have cloudlets a queue? What is scheduling policy? It is very important because the right scheduling policy can vastly reduce
mean response time without requiring the purchase of faster machines, but 





No hint about system stochastic components, like packet arrival process into the cloudlet.





Wearable system in the real-time health monitoring



Is not clear how they have modeled stochastic components of system, like packet arrival process into the cloudlet or their service time.
are they xponentially distributed, Uniformly distributed?





























\section{Possible improvement}













\subsection{RRRRRRRRRR}

\begin{quoting}[font=itshape, begintext={``}, endtext={''\cite[par.~3.1]{MSAReport}}]
It was shown that, via WiFi, the transmission power of a data packet of size 46 Bytes will cost about 30 mw and with a delay of 0.045 ms. On the other hand, a longer transmission range cellular network connection (e.g. 3G and LTE) is capable of transmitting the data packet to the cloud from any location that is cover by cellular network, which is usually a wider geographic area compared with the WiFi. It was shown that, via cellular, the transmission power of data packet of size 46 Bytes will cost about 300 mw and with a delay of 0.45 ms [...]


With WiFi, a WBAN user will be able to transmit the data packet to the cloud with low power and low delay compared with cellular technology, but with transmission range does not exceed 100 m. [...]


While the cellular connection is very costly in terms of power, transmission delay and connection cost (the WiFi is mostly free of charge),
\end{quoting}

\vspace{0.3cm}

According to \citet{MSAReport}, including many other researches cited by them, 

Is known that energy is one of the most valuable resources in a WBAN.

According to researchers of this study, the key method to reduce power consumption (and connection cost) is to develop a \textbf{self-adaptive system capable to change data communication technology dynamically when users move from one region to another}. 

To be more precise, if WiFi coverage is available, users will use WiFi technology to transmit a data packet, otherwise an user can use only cellular technology. If neither WiFi or cellular technology is available, users will buffer the packets until one of the above technologies is available.\cite[par.~3.1]{MSAReport}


\subsection{}

Unfortunately, researchers don't specify which version of the standard IEEE 802.11 is used in their simulations, however they declare that transmission range of WiFi base stations does not exceed 100 m. 

We believe that this system model can hardly be used for offloading in outdoor scenarios owing to short transmission range of WiFi stations and high economical effort to build proposed infrastructure.



In order to perform their simulations and study the impacts of cloudlet servers on packet delay and power consumptions, researchers model a large scale WBANs system using two monitored areas, one of $600$ x $400$ $m$ and $800$ x $800$ $m$, in which several users are moved in randomly. \citet{MSAReport} perform several simulations varying also the number of users and cloudlet.





 800 800 m = 

Researchers declare to study the impacts of installing cloudlet servers on the data collection performance in the monitoring area, the
following experiments were carried out. 400 users were moved in a random waypoint model within a monitored area of
600 400 m.


Although 802.11n and 802.11.ac data rates are comparable to or
even higher than the rates of the modern mobile networks, e.g. LTE, they


a 100 m transimmission range







The 802.11ah standard enables single-hop communication over distances up to 1000 m.



However, it is not suited for facilitating commu-
nication among a large number of IoT devices or for covering
large areas.











The power consumption of wireless transceivers is 
substantially impacted by their transmission power, which 
can be responsible for up to 70% of the total power 
consumption for off-the-shelf sensor nodes, e.g.

Energy is one of the most valuable resources in a 
WBAN. One of the major sources of energy waste in 
WBANs originates from internetwork interference between 
nearby WBANs which causes the Signal-to-Interference-
and-Noise Ratio (SINR) to drop and thereby throughput 
degrades.


----------

Roma - 1287,36 km2 = 1287360000 m2

2.011,5 cloudlet  


3.240.000




Medium Access
Control (MAC) protocols play a significant role in maximizing
the lifetime of WBAN by controlling the dominant sources
of energy waste, i.e., collision, idle listening, overhearing,
and control packet overhead.

To achieve communication over longer distances among a large number of low-power devices, several innovative concepts 

The 802.11ah standard enables single-hop communication over distances up to $1000 m$.


As expected, \citet{MSAReport} results demonstrate that increasing the number of users, for a given cloudlet deployment, will increase the power and delay of the collected data; this happens due an higher probability of collision and to interference between nearby WBAN devices, which causes the Signal-to-Interference-and-Noise Ratio (SINR) to drop and thereby throughput degrades.


 The
default medium access control (MAC) protocol for channel access in this standard is the distributed coordination
function (DCF), which is a random access scheme based
on carrier sense multiple access with collision avoidance
(CSMA/CA).

In addition to the basic access, an optional four way handshaking technique, known as request-to-send/clear-to-send
(RTS/CTS) mechanism has been standardized. Before transmitting a packet, a station operating in RTS/CTS mode “reserves”
the channel by sending a special Request-To-Send short frame.
The destination station acknowledges the receipt of an RTS
frame by sending back a Clear-To-Send frame, after which
normal packet transmission and ACK response occurs. Since
collision may occur only on the RTS frame, and it is detected
by the lack of CTS response, the RTS/CTS mechanism allows
to increase the system performance by reducing the duration
of a collision when long messages are transmitted. As an
important side effect, the RTS/CTS scheme designed in the
802.11 protocol is suited to combat the so-called problem of
Hidden Terminals


 The primary medium access control (MAC) technique of 802.11 is called distributed coordination function (DCF). DCF is a carrier sense multiple access
with collision avoidance (CSMA/CA) scheme with binary slotted
exponential backoff




the overheads associated with
the RTS and CTS transmissions – for short and time-critical data packets, this is not negligible 


\section{Medium Access Control}

The power consumption of wireless transceivers is substantially impacted by their transmission power, which can be responsible for up to \textbf{70\%} of the total power consumption. It was shown that the use of a Medium Access Control (MAC) protocols capable to minimize the dominant sources of energy waste like \textit{collision}, \textit{idle listening} and \textit{control packet overhead} is extremely useful to maximizing the lifetime of WBAN devices.

Unfortunately, \citet{MSAReport} don't give any details about used MAC protocol to transmit aggregated data packet to cloudlet, not even specifying the version of the standard IEEE 802.11 used in their simulations, although they declare that its transmission range does not exceed 100 m. Therefore, for our analysis we assume that \citet{MSAReport} have used legacy IEEE 802.11 protocol with default medium access control protocol for channel access, that is the distributed coordination function (DCF), which is a random access scheme based on carrier sense multiple access with collision avoidance (CSMA/CA).

In this section we try to prove that the use of traditional legacy IEEE 802.11 protocol can hardly be used for offloading in outdoor scenarios.

To achieve communication over longer distances among a large number of low-power devices, we can exploit the several innovative concepts introduced in the PHYsical (PHY) and Medium Access Control (MAC) layers of 802.11ah standard. As known, the 802.11ah standard enables single-hop communication over distances up to $1000 m$ and, thanks to Unlicensed < 1GHz band, better propagation characteristics in outdoor scenarios than traditional for Wi-Fi 2.4 and 5 GHz bands.

The Carrier Sense Multiple Access with Collision Avoid-ance (CSMA/CA)-based MAC of 802.11 fails to handle collisions, especially when thousands of devices simultaneously contend for a channel. To address this issue, the 802.11ah introduces a Restricted Access Window (RAW)-based mechanism that improves the network performance significantly. This mechanism splits the STAs into groups and allows the STAs belonging to a certain group to access the medium at any particular time frame.

Although .11ah PHY provides high data rates, the aggregate throughput may be very low because of the huge overhead inherent to short packet transmission typical for sensor networks.

o reduce it, the draft standard defines a new backward incompatible
format of shortened headers for data, management and novel control frames








\section{gd}

Re










nother approach is to reduce the
fidelity of the underlying algorithms when edge computing is
not available.




\section{Mobility management issue}

\section{Economical issues}

The lower the frequency, the better the penetration. Long waves can be transmitted through the oceans to a submarine while high frequencies can be blocked by a tree. The higher the frequency, the more the behaviour of the radio waves resemble that of light.

As known, the 802.11ah standard enables communication over distances up to $1000 m$ and, since it use a lower frequency (< 1GHz) it has better propagation characteristics in outdoor scenarios than traditional for Wi-Fi 2.4 and 5 GHz bands. 



\section{dsadas}




\subsection{Approximate computing techniques}

Everyday experience teaches us that battery technology is the very bottleneck of every mobile devices, WBAN devices included, and, unfortunately, as known battery technology is advancing very slowly if compared to the others computer technology, like CPU, storage systems and RAM.

A promising solution is \textbf{approximate computing} (and \textbf{storage}), which is based on the intuitive observation that while performing exact computation require high amount of resources, allowing selective approximation or imperfect computations results can provide drastic energy savings.

Today many important mobile applications use a some kind of approximation computing technique, including computer vision, sensor data analysis, machine learning, augmented reality and signal processing. 

Approximate computing leverage the presence of error-tolerant code regions in applications and per-
ceptual limitations of users to intelligently trade off implementation, storage and/or
result accuracy for performance or energy gains. In brief, AC exploits the gap between
the level of accuracy required by the applications/users and that provided by the com-
puting system, for achieving diverse optimizations.



\section{Possible improvements}

\subsubsection{Unikernel instead of }

As stated by \citet{MSAReport}, cloudlet are intended to run a certain number of virtual machine, running x86 Linux, over Xen hypervisor.

It is shown, that running a fully functional operating system over a virtual machine 

General purpose operating system like Linux or Windows were designed to be run on hardware, so they have all the complexity needed for a variety of hardware drivers from an assortment of vendors with different design concepts. These operating systems are also intended to be multi-user, multi-process, and multi-purpose. They are designed to be everything for everyone, so they are necessarily complex and large.




\bibliographystyle{ACM-Reference-Format}
\bibliography{Bibliography}

\appendix

\end{document}
\endinput
