\documentclass[10pt]{beamer}
%\usetheme[
%%% option passed to the outer theme
%    progressstyle=fixedCircCnt,   % fixedCircCnt, movingCircCnt (moving is deault)
%]{}
  
% If you want to change the colors of the various elements in the theme, edit and uncomment the following lines

% Change the bar colors:
%\setbeamercolor{Feather}{fg=red!20,bg=red}

% Change the color of the structural elements:
%\setbeamercolor{structure}{fg=red}

% Change the frame title text color:
%\setbeamercolor{frametitle}{fg=blue}

% Change the normal text color background:
%\setbeamercolor{normal text}{fg=black,bg=gray!10}

%-------------------------------------------------------
% INCLUDE PACKAGES
%-------------------------------------------------------

\usepackage[utf8]{inputenc}
\usepackage{quoting}
\usepackage{tikz}
\usepackage{natbib}

%-------------------------------------------------------
% DEFFINING AND REDEFINING COMMANDS
%-------------------------------------------------------

\usetheme{Warsaw}

%-------------------------------------------------------
% INFORMATION IN THE TITLE PAGE
%-------------------------------------------------------

\title[] % [] is optional - is placed on the bottom of the sidebar on every slide
{ % is placed on the title page
      \textbf{Analysis of "Cloudlet-based Efficient Data Collection in Wireless Body Area Networks" report}
}

\subtitle[]
{
      \textbf{MSA 2019-2020}
}

\author[Andrea Graziani - 0273395]
{      Andrea Graziani - 0273395 \\
      {}
}

\institute[]
{
      Università degli Studi di Roma “Tor Vergata” \\
      FACOLTA' DI INGEGNERIA \\
      Corso di Laurea Magistrale in Ingegneria Informatica
  
  %there must be an empty line above this line - otherwise some unwanted space is added between the university and the country (I do not know why;( )
}

\date{\today}

%-------------------------------------------------------
% THE BODY OF THE PRESENTATION
%-------------------------------------------------------

\begin{document}

%-------------------------------------------------------
% THE TITLEPAGE
%-------------------------------------------------------

{% % this is the name of the PDF file for the background
\begin{frame}[plain,noframenumbering] % the plain option removes the header from the title page, noframenumbering removes the numbering of this frame only
  \titlepage % call the title page information from above
\end{frame}}


\section{Introduction}
\subsection{Research's goal}
%---------------------------------------------------------------------------------------------------------%
\begin{frame}{Introduction and research's goal}
%---------------------------------------------------------------------------------------------------------%

\begin{itemize}

\item What is meant by WBAN (\textbf{Wireless Body Area Networks})?
\item What are the \textbf{objectives} of \citet{MSAReport} research?

\vspace{0.3cm}

\begin{quoting}[font=itshape, begintext={``}, endtext={''\cite[par.~1.4]{MSAReport}}]
The main goal of this paper is to develop a large scale WBANs system in the presence of cloudlet-based data collection model. The objective is to minimize end-to-end packet cost by dynamically choosing data collection to the cloud by using cloudlet based system [...] reducing packet-to-cloud energy, the proposed work also attempt to minimize the end-to-end packet delay.
\end{quoting}

\vspace{0.3cm}

\item In other words, the goal of their work is to built an \textit{efficient} WBANs exploiting \textbf{edge computing paradigm}, in order to \textbf{minimize average power consumption} and \textbf{transmission delay} when \textbf{Personal Digital Assistant} (\textbf{PDA}) of any WBAN user transmits collected data.
\end{itemize}

%---------------------------------------------------------------------------------------------------------%
\end{frame} 
%---------------------------------------------------------------------------------------------------------%
\subsection{Analysis of existing WBANs implementations}
%---------------------------------------------------------------------------------------------------------%
\begin{frame}{Analysis of existing WBANs implementations} 
%---------------------------------------------------------------------------------------------------------%

\begin{itemize}
\item Why \citet{MSAReport} state that existent solutions aren't optimal for WBANs applications?

\begin{enumerate}
\item They are based on cloud computing paradigm which has several disadvantages, the most important of which is \textbf{latency} problems. 
\item They exploit \textbf{cellular network} as communication technology.

\begin{quoting}[font=itshape, begintext={``}, endtext={''\cite[par.~3.1]{MSAReport}}]
It was shown that, via WiFi, the transmission power of a data packet of size 46 Bytes will cost about 30 mw and with a delay of 0.045 ms. On the other hand, a longer transmission range cellular network connection (e.g. 3G and LTE) is capable of transmitting the data packet to the cloud from any location that is cover by cellular network, which is usually a wider geographic area compared with the WiFi. It was shown that, via cellular, the transmission power of data packet of size 46 Bytes will cost about 300 mw and with a delay of 0.45 ms.
\end{quoting}

\end{enumerate}

\end{itemize}

%---------------------------------------------------------------------------------------------------------%
\end{frame} 
%---------------------------------------------------------------------------------------------------------%
\section{An Edge-Accelerated, Cloud-Native Application}
\subsection{The role of communication technology}
%---------------------------------------------------------------------------------------------------------%
\begin{frame}{The role of communication technology} 
%---------------------------------------------------------------------------------------------------------%

\begin{itemize}

\item According to researchers, the role of communication technology is \textbf{critical for energy consumption reduction}:
\begin{itemize}
\item all research is based on the observation according to which using WiFi communication technology a WBAN user is able to transmit data packet to the enterprise cloud with \textbf{low power consumption}, \textbf{low delay} (\textbf{not always!}) and mostly \textbf{no connection cost} compared with cellular communication technology. 
\end{itemize}

\item How to exploit WiFi technology in WBAN systems taking into account its \textbf{short transmission range} compared to cellular network?
\begin{itemize}
\item Providing an edge computing infrastructure made up of several cloudlet geographically distributed and equipped with computational, storage and communication capabilities.
\end{itemize}
\end{itemize}


%---------------------------------------------------------------------------------------------------------%
\end{frame} 
%---------------------------------------------------------------------------------------------------------%
\subsection{The role of edge computing}
%---------------------------------------------------------------------------------------------------------%
\begin{frame}{The role of edge computing} 
%---------------------------------------------------------------------------------------------------------%

\begin{itemize}

\item According to researchers, the role of communication technology is \textbf{critical for transmission delay reduction}:

\begin{itemize}
\item deploying cloudlet \textit{close} to final users, thanks to \textbf{high-bandwidth single-hop connection}, is possible to reduce RTT and to increase end-to-end bandwidth; in other words, we can achieve \textbf{network-proximity}.\citep{TheSeminalRoleEdgeNativeApplications}

\item However, is very important to not confuse \textit{physical proximity} with \textit{network-proximity} (\textbf{!!!})
\end{itemize}

\end{itemize}

%---------------------------------------------------------------------------------------------------------%
\end{frame} 
%---------------------------------------------------------------------------------------------------------%
%---------------------------------------------------------------------------------------------------------%
\begin{frame}{The role of edge computing} 
%---------------------------------------------------------------------------------------------------------%

\begin{itemize}

\item Be aware that researchers solution is \textit{still} using \textit{cloud} resources.

\begin{itemize}
\item Cloud plays a very important role in this kind of system, because its resources are exploited to overcome WBAN devices constraints.
\item Cloud represents, in terms of archival preservation, the \textit{safest place to store data}, ensuring the long-term integrity and accessibility, and the \textit{main execution site to perform expensive computations}, thanks to its almost unlimited \textbf{elasticity}.
\end{itemize}

\end{itemize}

%---------------------------------------------------------------------------------------------------------%
\end{frame} 
%---------------------------------------------------------------------------------------------------------%
%---------------------------------------------------------------------------------------------------------%
\begin{frame}{The role of edge computing} 
%---------------------------------------------------------------------------------------------------------%

\vspace{0.3cm}

\begin{quoting}[font=itshape, begintext={``}, endtext={''\cite[par.~3.2]{MSAReport}}]
In our implementation, the enterprise cloud system will be the ultimate destination of the collected data. The enterprise cloud is also able to send messages back to both the cloudlet system or to WBAN users.
\end{quoting}

\vspace{0.3cm}

\begin{quoting}[font=itshape, begintext={``}, endtext={''\cite[par.~4.1]{MSAReport}}]
The enterprise cloud system is a centralized management and storage point that can be accessed by different organizations that are interested in a certain type of data. Another important feature of the cloudlet system is the ability of bidirectional communications between many WBANs users.
\end{quoting}

\vspace{0.3cm}

%---------------------------------------------------------------------------------------------------------%
\end{frame} 
%---------------------------------------------------------------------------------------------------------%
%---------------------------------------------------------------------------------------------------------%
\begin{frame}{The role of edge computing} 
%---------------------------------------------------------------------------------------------------------%

\begin{itemize}

\item \citet{MSAReport} have built a system, capable to exploit edge computing resources, intended for so called \textit{edge-accelerated, cloud-native applications}. \cite{TheSeminalRoleEdgeNativeApplications}\cite{TheEmergenceOfEdgeComputing} In fact:

\begin{description}

\item[cloud-native] because, to overcome stringent constraints on battery, performance and storage of WBAN devices, cloud resources are exploited, through offloading techniques over a wireless network, \textbf{to execute critical task}; in other words, as stated also by \citet{MSAReport}, \textit{cloud is the primary execution site and it is essential to provide services to final users}.

\item[edge-accelerated] because the system exploits edge computing resources \textbf{only when available}. In that way, that system is capable to provide optimal performance, but \textit{edge resources represents a secondary execution site, that is they are optional to provide services}.

\end{description}

\end{itemize}

%---------------------------------------------------------------------------------------------------------%
\end{frame} 
%---------------------------------------------------------------------------------------------------------%
\subsection{A self-adaptive system}
%---------------------------------------------------------------------------------------------------------%
\begin{frame}{A self-adaptive system} 
%---------------------------------------------------------------------------------------------------------%

\begin{itemize}
\item \citet{MSAReport} implemented a \textit{self-adaptive behaviour} according to which \textbf{communication technology is dynamically changed} when users move from one region to another. 

\item \citet{MSAReport} have identified three different regions:\cite[par.~3.1]{MSAReport}
\begin{description}

\item[Cloudlet Region (CR)] where WiFi coverage is available, so a user can use it to transmit data to the cloudlet.

\item[Enterprise Region (ER)] where only cellular coverage is available, therefore  a user can use only cellular technology to transmit a data packet to cloud. 

\item[Not-covered Region (NC)] where neither WiFi nor cellular technology is available. In this case a user should buffer the packets until one of the above technologies is available, then to be able of transmitting the packet to the enterprise cloud. 

\end{description}

\item In other words, WBAN users are capable to exploit edge computing resources if and only if they are in a \textit{Cloudlet Region}, otherwise they go on transmitting data directly to the cloud. (in fact, that solution is edge-accelerated, cloud-native application)

\end{itemize}

%---------------------------------------------------------------------------------------------------------%
\end{frame} 
%---------------------------------------------------------------------------------------------------------%
%---------------------------------------------------------------------------------------------------------%
\begin{frame}{A self-adaptive system} 
%---------------------------------------------------------------------------------------------------------%

\begin{itemize}

\item From a deployment point of view, this system represents a \textbf{distributed self-adaptive system} since a self-adaptive software, including both \textit{managed software} and \textit{managing software}, is deployed in every WBAN users device.\cite{PatternsDecentralizedSelf}

\item From decisions control point of view, we can consider it a \textbf{decentralized self-adaptive system} too due to the lack of a central control component that decides about when and how to perform an adaptation.\cite{PatternsDecentralizedSelf} 

However, is very important to precise that in \citet{MSAReport} system there is \textbf{no coordination} between WBAN nodes referring to control decision. All  of them act independently, deciding adaptations autonomously, therefore, \textit{any kind of control decisions decentralization pattern is been used}. 

\end{itemize}

%---------------------------------------------------------------------------------------------------------%
\end{frame} 
%---------------------------------------------------------------------------------------------------------%
%---------------------------------------------------------------------------------------------------------%
\begin{frame}{A self-adaptive system} 
%---------------------------------------------------------------------------------------------------------%
 
\begin{itemize}
\item From an architectural point of view, that system uses a \textbf{top-down approach} based on \textbf{MAPE-K feedback control loop}. Researchers wrote that:

\vspace{0.3cm}

\begin{quoting}[font=itshape, begintext={``}, endtext={''\cite[par.~5.1]{MSAReport}}]
each VC region is able to serve a certain number of users [...] Then, the extra users within the VCs have to send the data via the cellular communication, even though, they are within the VCs. 
\end{quoting}

\vspace{0.3cm}

\item In other words, \textit{WBAN devices exhibit a behavior according to which they can decide to offload towards cloud instead of cloudlet if Wifi network is highly congested}. This point is very important because an highly congested WiFi can lead to high latency, therefore, in accordance with policies established by \citet{MSAReport}, WBAN nodes  send data cellular communication, even though, they are within a cloudlet region.  

\end{itemize}

%---------------------------------------------------------------------------------------------------------%
\end{frame} 
%---------------------------------------------------------------------------------------------------------%
%---------------------------------------------------------------------------------------------------------%
\begin{frame}{A self-adaptive system} 
%---------------------------------------------------------------------------------------------------------%

\begin{description}

\item[Knowledge/Goal] The goals established by \citet{MSAReport} are average power consumption and average transmission delay minimization.

\item[Monitor] The monitoring process is responsible for collecting data from environment like network quality and WiFi or cellular connection availability. 

\item[Analyse] Measured data are analysed in order to detect WiFi congestion condition level ("Low" or "High") and check what is the current region ("NC", "ER" or "CR") where user is.

\item[Plan] Select and plan a solution to achieve established goal in according to system state diagram (\textbf{see the report!})

\item[Execute] Change communication technology according to current state.

\end{description}

%---------------------------------------------------------------------------------------------------------%
\end{frame} 
%---------------------------------------------------------------------------------------------------------%
%---------------------------------------------------------------------------------------------------------%
\begin{frame}{A self-adaptive system} 
%---------------------------------------------------------------------------------------------------------%

\begin{itemize}

\item Finally, \citet{MSAReport} solution can be considered as \textbf{application-transparent} since WBAN applications are \textit{not} informed that a communication technology change is occurred; in other words, underlying system is the only responsible for the adaptation. In this way, compatibility with already existing applications is assured, although adaptations can affect negatively on WBAN applications. 

\end{itemize}


%---------------------------------------------------------------------------------------------------------%
\end{frame} 
%---------------------------------------------------------------------------------------------------------%
\subsection{The use of cyber-foraging techniques}
%---------------------------------------------------------------------------------------------------------%
\begin{frame}{The use of cyber-foraging techniques} 
%---------------------------------------------------------------------------------------------------------%

\begin{itemize}
\item Obliviously, WBAN devices cannot have enough computation and storage resources to manage that amount of data due to their very strictly constrains in term of size and power consumption. In order to overcome limitations of mobile devices used in WBAN, \citet{MSAReport} adopt several \textbf{cyber-foraging techniques}.

\item \citet{MSAReport} have adopted \textbf{data staging} tactic too in order to improve WBAN applications performances managing \textbf{field-collected data}. In fact, when a WBAN application complete all its data collection operations, following situations may happen:

\begin{itemize}
\item If a WBAN user is close to a cloudlet, WBAN devices offload data to cloudlet and, when the operation is complete, they delete transmitted data to free up storage space. In addition, cloudlet forwards to enterprise cloud all data that was collected by the multiple users.

\item If a WBAN user cannot establish a connection with a cloudlet and a cellular data connection is available, all collected data will be offloaded directly to the cloud.
\end{itemize}



\end{itemize}

%---------------------------------------------------------------------------------------------------------%
\end{frame} 
%---------------------------------------------------------------------------------------------------------%
%---------------------------------------------------------------------------------------------------------%
\begin{frame}{The use of cyber-foraging techniques} 
%---------------------------------------------------------------------------------------------------------%

\begin{itemize}

\item In other words, is clear that external resources are located in both \textbf{single-hop} and \textbf{multi-hop} proximity of the mobile devices that use them. In fact, when collected data are offloaded to a remote resource, that is enterprise cloud, \textbf{synchronous operations with multiple network hops} between the mobile device and the cloud are involved while, when data are offloaded to cloudlet, a \textbf{single-hop connection} is involved (although is not clear if cloudlet need to be connected at runtime to cloud or not).

\end{itemize}

%---------------------------------------------------------------------------------------------------------%
\end{frame} 
%---------------------------------------------------------------------------------------------------------%
%---------------------------------------------------------------------------------------------------------%
\begin{frame}{The use of cyber-foraging techniques} 
%---------------------------------------------------------------------------------------------------------%

\begin{itemize}
\item \citet{MSAReport} adopt a very static approach about offloading operations because, if a offload target is available, data will be \textbf{always} offloaded. 

\item This decision can lead to a negative consequence to \textbf{reduce resource efficiency} because, even though executing the expensive computation on a surrogate leads to energy efficiency, \textit{changing network conditions might cause greater resource consumption}: if a WBAN node detects that WiFi network conditions are bad and decide to use cellular network, inevitably more energy will be consumed increasing transmission delay.

\end{itemize}

%---------------------------------------------------------------------------------------------------------%
\end{frame} 
%---------------------------------------------------------------------------------------------------------%
\subsubsection{The cloudlet discovery and provisioning issue}
%---------------------------------------------------------------------------------------------------------%
\begin{frame}{The cloudlet discovery and provisioning issue} 
%---------------------------------------------------------------------------------------------------------%

We believe that the cyber-foraging system built by \citet{MSAReport} is too simple to achieve research's goals, because it lacks of some very important features. In fact, it is shown that cyber-foraging systems have \textit{at a minimum} the following combination of functional requirements \cite{DecisionModel}:

\begin{itemize}

\item  A need for computation offload, data staging, or both.
\item  A need to \textbf{provision a surrogate} with the offloaded computation or data staging capabilities.
\item  A need for the mobile device to \textbf{locate a surrogate} at runtime.

\end{itemize}

%---------------------------------------------------------------------------------------------------------%
\end{frame} 
%---------------------------------------------------------------------------------------------------------%
%---------------------------------------------------------------------------------------------------------%
\begin{frame}{Cloudlet discovery issue} 
%---------------------------------------------------------------------------------------------------------%

\begin{itemize}

\item Although WBAN users \textit{need} to be able to locate available cloudlet in an area where stage data, \textbf{cloudlet discovery} issue has \textbf{not} been addressed by \citet{MSAReport}

\item \textbf{cloudlet discovery} is very important because affects negatively energy consumption and response time and, anyway, it is a \textbf{functional requirements} for any cyber-foraging systems. 

\begin{itemize}
\item The problem of \textbf{selection algorithm} to select best cloudlet.
\item The problem of \textbf{security}.
\end{itemize}

\item Possible solution? \textbf{Cloud surrogate directory tactic!}

\end{itemize}

%---------------------------------------------------------------------------------------------------------%
\end{frame} 
%---------------------------------------------------------------------------------------------------------%
%---------------------------------------------------------------------------------------------------------%
\begin{frame}{Cloudlet provisioning issue} 
%---------------------------------------------------------------------------------------------------------%

\begin{itemize}

\item In the same way, \textbf{surrogate provisioning} is another functional requirements for any cyber-foraging systems which is \textbf{not} addressed by \citet{MSAReport}, therefore is not clear how cloudlets manage offloaded computation and/or data processing operations.

\item We will see later that \citet{MSAReport} had shown that \textbf{packet process delay} is affected by the number of virtual machine deployed in cloudlet. However, provisioning tactic affects this process delay too. It is shown that \textit{pre-provisioned cloudlet tactic} have the advantage of shorter provisioning times because the capabilities already reside on cloudlet, providing shorter response times to requests from mobile devices.\cite{DecisionModel}

\end{itemize}


%---------------------------------------------------------------------------------------------------------%
\end{frame} 
%---------------------------------------------------------------------------------------------------------%
\subsection{Communication style}
%---------------------------------------------------------------------------------------------------------%
\begin{frame}{Communication style} 
%---------------------------------------------------------------------------------------------------------%

\begin{itemize}

\item When a device cannot transfer data packet to cloudlet for some reason, WBAN users are obliged to communicate directly with server-side application running on cloud, therefore a classical \textbf{client-server style communication} is adopted, which use classic \textit{point-to-point} and \textit{synchronous} communications.

\item When a WBAN user is capable to communicate through a server, the system adopts a \textbf{loose-connector} communication style: \textbf{server-side client/agent/server model}. In that case, the \textit{agent} is the cloudlet, which, acting as a surrogate to perform cyber-foraging tactics, is located on the so-called "\textit{fixed}" side portion of the network and, precisely, on the edge of the network. 

\end{itemize}

%---------------------------------------------------------------------------------------------------------%
\end{frame} 
%---------------------------------------------------------------------------------------------------------%

\section{Simulations Analysis}
\subsection{First experiment set}

%---------------------------------------------------------------------------------------------------------%
\begin{frame}{First experiment set}
%---------------------------------------------------------------------------------------------------------%


First experiment set goal is to quantify both \textbf{packet process delay} and \textbf{power consumption}, due to packet computation, by cloudlet varying following system parameter.

\begin{enumerate}

\item number of virtual machine deployed on a cloudlet (0, 2, 4 or 8). The maximum number of deployable virtual machines is bounded by the number of physical processors available on cloudlet (\citet{MSAReport} prototype cloudlet have 8 physical processors, therefore 8 is the maximum number of deployable VMs).
 
\item processing speed of data packet, from a minimum 100 to a maximum of 900 \textbf{million instructions per second} (MIPS).

\item number of WBAN users (up to 150 users).

\end{enumerate}


%---------------------------------------------------------------------------------------------------------%
\end{frame} 
%---------------------------------------------------------------------------------------------------------%
%---------------------------------------------------------------------------------------------------------%
\begin{frame}{First experiment set: Results}
%---------------------------------------------------------------------------------------------------------%

\begin{itemize}

\item \textbf{Increasing the number of virtual machines running on a cloudlet, packet process delay will be reduced.}

It is shown that, if data packets can be processed in \textbf{parallel}, the \textbf{blocked time} can be reduced. In fact, fixed data packet \textit{average arrival rate} $\lambda$ and \textit{average service rate} $\mu$ of each virtual machine, system cloudlet \textbf{utilization}, that is \textit{the fraction of time according to which cloudlet is busy}, is reduced; each virtual machine, by symmetry, sees an arrival rate of $\dfrac{\lambda}{k}$, where $k$ is the number of deployed virtual machines. Therefore cloudlet utilization, which is equal to $\dfrac{\lambda}{k\mu}$, decrease increasing $k$. It is shown that, decreasing utilization, \textbf{blocked time} decreases.\cite{BassSoftwareArchitecture2003}

\end{itemize}

%---------------------------------------------------------------------------------------------------------%
\end{frame} 
%---------------------------------------------------------------------------------------------------------%
%---------------------------------------------------------------------------------------------------------%
\begin{frame}{First experiment set: Results}
%---------------------------------------------------------------------------------------------------------%

\begin{itemize}

\item \textbf{Fixed the average time required to process a packet data on a CPU, that is for a given MIPS process speed, power consumption and processing delay are increased by increasing the number of users.} 

This happens because cloudlet utilization increases because $\lambda$ is higher, therefore the fraction of time according to which cloudlet is busy increases too; so, since more time is needed to complete user tasks, more power is consumed.

\end{itemize}

%---------------------------------------------------------------------------------------------------------%
\end{frame} 
%---------------------------------------------------------------------------------------------------------%
\subsection{Second experiment set}
%---------------------------------------------------------------------------------------------------------%
\begin{frame}{Second experiment set} 
%---------------------------------------------------------------------------------------------------------%

Second experiment set was carried out in order to quantify advantages of using edge computing resources. To be more precise, that simulation monitors the effects on \textbf{average transmission power and delay} of data packed send by users PDA to cloudlet, varying following system parameters:

\begin{enumerate}
\item number of cloudlet deployed (from 0 to 6).
\item number of WBAN users (set to be 400, 600, 800, 1000, 1200 and 1400).
\end{enumerate}

Monitored area size is fixed to $600 \times 400\;m$.

%---------------------------------------------------------------------------------------------------------%
\end{frame} 
%---------------------------------------------------------------------------------------------------------%
%---------------------------------------------------------------------------------------------------------%
\begin{frame}{Second experiment set: Results}  
%---------------------------------------------------------------------------------------------------------%

\begin{itemize}

\item As expected, increasing the number of available cloudlets in monitored area, average transmission delay of data packed is reduced. It happens since the probability  that an user is close to one of them increases. Having more opportunities to transmit data within cloudlet coverage, users can benefit of \textit{cloudlet network proximity}. 

In fact, when cloudlet coverage is available, data is offloaded to a cloudlet using a high bandwidth single-hop connection, which decrease RTT. Conversely, when cloudlet coverage isn't available, data is offloaded to cloud using a multi-hop network connections which involves high RTT and likely low bandwidth connection.\cite{ArchitecturalTacticsCyberForaging}

\end{itemize}

%---------------------------------------------------------------------------------------------------------%
\end{frame} 
%---------------------------------------------------------------------------------------------------------%
%---------------------------------------------------------------------------------------------------------%
\begin{frame}{Second experiment set: Results} 
%---------------------------------------------------------------------------------------------------------%

\begin{itemize}

\item Increasing the number of available cloudlets, average transmission power of data packed is reduced too. It is due to the use of WiFi technology which is, as already said, more energy-efficient compared to cellular network.

\end{itemize}

%---------------------------------------------------------------------------------------------------------%
\end{frame} 
%---------------------------------------------------------------------------------------------------------%
%---------------------------------------------------------------------------------------------------------%
\begin{frame}{Second experiment set: Results} 
%---------------------------------------------------------------------------------------------------------%

\begin{itemize}

\item Increasing the number of users, both average transmission power and delay increase. This happens because:

\begin{enumerate}
\item When a cloudlet area contains a large number of users, interferences increase, leading to higher error bit rate and affecting negatively both transmission delay, due to data packets or acknowledgements loss, and power, due to packet retransmission.

\item Since a polling MAC scheme is used, an higher number of users within an cloudlet WiFi coverage area increases the time needed to pool every node by AP. 

\item As stated by \citet{MSAReport}, an high congested WiFi network can cause WBAN devices to perform a self-adaptation action according to which communication technology is switched, using cellular network to send packet data instead WiFi, affecting negatively aforementioned metrics.
\end{enumerate}

\end{itemize}

%---------------------------------------------------------------------------------------------------------%
\end{frame} 
%---------------------------------------------------------------------------------------------------------%
\subsection{Third experiment set}
%---------------------------------------------------------------------------------------------------------%
\begin{frame}{Third experiment set}
%---------------------------------------------------------------------------------------------------------%

The last experiment set is focusing on monitoring aforementioned performance metric varying cloudlet geographical placement. Monitoring an $800 \times 800 \; m$ area, simulations were carried out varying following system parameters:

\begin{enumerate}
\item number of cloudlet deployed (up to 16 cloudlets).
\item number of WBAN users (up to 1400 users).
\item WBAN user's positions (same parameters as before)
\item cloudlet geographical placement (using very different patterns, classified in three categories by \citet{MSAReport}: \textit{Adjacent}, \textit{Distant} and \textit{Intermediate})
\end{enumerate}

%---------------------------------------------------------------------------------------------------------%
\end{frame} 
%---------------------------------------------------------------------------------------------------------%
%---------------------------------------------------------------------------------------------------------%
\begin{frame}{Third experiment set: Results}
%---------------------------------------------------------------------------------------------------------%

Experiments results are the following:

\begin{itemize}

\item As expected, independently from cloudlet deployment pattern, increasing the number of cloudlet, the impact of cloudlet geographical placement on average transmission power and delay is negligible since, in that way, the opportunities to send the data packet to a cloudlet, using WiFi and with minimum cost of power and delay, increase.

\item Fixed cloudlet and users number, deploying cloudlet using an intermediate category pattern, that is placing cloudlet neither too far apart nor too close, system performance are better than other patterns belonging to other categories.

\end{itemize}
\end{frame}

% ************************************************************************** %

\begin{frame}[plain,noframenumbering]
  Grazie per l'attenzione!
\end{frame}
\bibliography{Bibliography}


\end{document}